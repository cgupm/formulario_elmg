\documentclass[12pt,a4paper]{article}
\usepackage{amsmath}    % need for subequations
\usepackage{hyperref}   % use for hypertext links, including those to external documents and URLs
\usepackage{esint}
\usepackage{amsfonts}
\usepackage[utf8]{inputenc}
\usepackage{makeidx}
\usepackage{amsfonts}
\usepackage{fullpage}
\usepackage[spanish]{babel}


\title{Formulario de ELMG - ETSIT UPM}
\author{Carlos García-Mauriño}
\date{\today}

\begin{document}

\maketitle

\twocolumn

\section{Introducción}
\label{sec:introduccion}

\subsection{Electromagnetismo}
\label{sub:electromagnetismo}

\subsubsection{Densisdades o distribuciones de corriente}
\label{ssub:densisdades_o_distribuciones_de_corriente}

\begin{itemize}
		\item \textbf{Superficiales:}
				\[ I_{TOT} = \int_L\vec{J_s} \cdot \hat{n} \cdot dl \]
				\[ \vec{J_s} = \rho_s \cdot \vec{v} \]
\end{itemize}

\subsubsection{Campo electromagnetico}
\label{ssub:campo_electromagnetico}

\begin{itemize}
		\item \textbf{Fuerza de Lorentz}:
				\[ \vec{F} = q \cdot \vec{E} + q \cdot ( \vec{v} \times \vec{B} ) \]
		\item \textbf{Vector de Poynting}:
				\[ \vec{P} = \vec{E} \times \vec{H} \quad \left( \frac{W}{m^2} \right) \]
\end{itemize}

\subsubsection{Condiciones de contorno}
\label{ssub:condiciones_de_contorno}

\[ \vec{n}_{12} \cdot ( \vec{D}_2 - \vec{D}_1 ) = \rho_s \]

\subsubsection{Ecuación de continuidad y ley de conservación de la carga}
\label{ssub:ecuacion_de_continuidad_y_ley_de_conservacion_de_la_carga}

Una corriente estacionaria tiene divergencia 0 al no haber variación temporal.

\[ \nabla \cdot \vec{J_c} + \frac{\partial \rho_v}{\partial t} = 0\]

\section{Electrostática}
\label{sec:electrostatica}

\subsection{Introducción}
\label{sub:introduccion}

\begin{itemize}
		\item No hay varaciación temporal.
		\item No hay movimiento de cargas $ \vec{J_c} = 0 $ 
\end{itemize}

\subsubsection{Ecuaciones de Maxwell}
\label{ssub:ecuaciones_de_maxwell}

\[
\begin{aligned}
\nabla\times\vec{E} \quad & = \quad -\frac{\partial\vec{B}}{\partial t},  & \quad \text{(Ley de Ampère)}   \\[5pt]
\nabla\times\vec{H} \quad & = \quad \vec{J_c} + \frac{\partial\vec{D}}{\partial t},   & \quad \text{(Ley de Faraday)} \\[5pt]
\nabla\cdot\vec{D} \quad & = \quad \rho_v,  & \quad \text{(Ley de Coulomb)} \\[5pt]
\nabla\cdot\vec{B} \quad & = \quad 0. & \quad \text{(Ley de Gauss)}
\end{aligned}
\]

\subsubsection{Ley de Ohm generalizada}
\label{ssub:ley_de_ohm_generalizada}

\[ \vec{J_c} = \sigma \cdot \vec{E} \]

\subsection{Potencial eléctrico}
\label{sub:potencial_electrico}

\[ \vec{E} = - \nabla \phi \]
\[ \phi = - \int \vec{E} \cdot d \vec{l} \]

\subsection{Cálculo de $ \vec{E} $ y $ \phi $ }
\label{sub:calculo_de_e_y_phi_}

\subsubsection{Aportaciones infnitesimales}
\label{ssub:aportaciones_infnitesimales}

\begin{itemize}
		\item \textbf{Cargas puntuales}:
				\[ \vec{E}( \vec{r} ) = \frac{q \cdot ( \vec{r} - \vec{r}\,' )}{4 \pi \varepsilon \cdot | \vec{r} - \vec{r}\,' |^3} \quad (V/m) \]
				\[ \phi ( \vec{r} ) = \frac{q}{4 \pi \varepsilon \cdot | \vec{r} - \vec{r}\,' |} \quad (V) \]
\end{itemize}

\subsubsection{Ley de Gauss}
\label{ssub:ley_de_gauss}

\[ \oiint\limits_{S} \vec{D} \cdot d \vec{S} =  Q_{encerrada} \]

\subsubsection{Ley de Poisson para medios homogéneos}
\label{ssub:ley_de_poisson_para_medios_homogeneos}

\[ \Delta \phi = - \frac{\rho_V}{ \varepsilon } \]

\subsection{Energía y densidad de energía electrostática}
\label{sub:energia_y_densidad_de_energia_electrostatica}

\[ W_e = \frac{1}{2} \iiint_V \varepsilon \cdot \vec{E}^2 \, dV \quad (J) \]

\subsection{Condensadores}
\label{sub:condensadores}

\[ C = \frac{Q}{V_0} \quad (F) \]

\subsubsection{Energía almacenada en un condensador}
\label{ssub:energia_almacenada_en_un_condensador}

\[ W_e = \frac{1}{2} C \cdot V_0^2 = \frac{Q^2}{2 \cdot C} \quad (J) \]

\subsection{Dipolo eléctrico}
\label{sub:dipolo_electrico}

\[ Q_{total} = 0 \]

\[ \vec{E} ( \vec{r} ) = \frac{1}{4 \pi \varepsilon \cdot r^3} \left[ \frac{3 \cdot ( \vec{p} \cdot \vec{r} ) \cdot \vec{r}}{r^2} - \vec{p} \right] \quad (V/m) \]
\[ \phi ( \vec{r} ) = \frac{ \vec{p} \cdot \vec{r}}{4 \pi \varepsilon \cdot r^3} \quad (V) \] 

\subsubsection{Calculo de $ \vec{p} $ }
\label{ssub:calculo_de_p}

\[ \vec{p} = \iiint_{V'} \vec{r}\,' \cdot \rho_{V'} \cdot dV' \quad (C \cdot m) \]
\[ \vec{p} = q \cdot d \cdot \vec{u}_{-+} \quad (C \cdot m) \]


\section{Magnetostática}
\label{sec:magnetostatica}

\subsection{Potencial vector magnético $ \vec{A} \quad \left( \frac{Wb}{m} \right) $ }
\label{sub:potencial_vector_magnetico}

En magnetostática cumple:

\[ \vec{B} = \nabla \times \vec{A} \]
\[ 0 = \nabla \cdot \vec{A} \]

\[ \Delta \vec{A} = - \mu \vec{J} \]

\subsection{Calculo de $ \vec{B} $ }
\label{sub:calculo_de_b_}

\subsubsection{Ley de Biot-Savart}
\label{ssub:ley_de_biot_savart}

\[ \vec{B} ( \vec{r} ) = \frac{\mu}{4 \pi} \iiint_{V'} \frac{ \vec{J}_{V'} \times ( \vec{r} - \vec{r}\,' )}{| \vec{r} - \vec{r}\,' |^3} dV' \]
\[ \vec{B} ( \vec{r} ) = \frac{\mu I}{4 \pi} \cdot \frac{ \vec{dl}\,' \times ( \vec{r} - \vec{r}\,' )}{| \vec{r} - \vec{r}\,' |^3} \]

\subsubsection{Ley de Ampère}
\label{ssub:ley_de_ampere}

\[ \oint_C \vec{H} \cdot \vec{dl} = I \]

\subsection{Momento dipolar magnético $ \vec{m} $ }
\label{sub:momento_dipolar_magnetico_m}

Para puntos lejanos:

\[ \vec{B} ( \vec{r} ) = \frac{\mu}{4 \pi r^3} \left[ \frac{3 ( \vec{m} \cdot \vec{r} ) \cdot \vec{r}}{ r^2} - \vec{m} \right] \quad (T) \]

\subsubsection{Cálculo de $ \vec{m} $ }
\label{ssub:calculo_de_m}

\[ \vec{m} = \frac{1}{2} \iiint_{V'} \left[ \vec{r}\,' \times \vec{J}_{V'} ( \vec{r}\,' ) \right] dV' \]
\[ \vec{m} = \frac{1}{2} \int_{L'} I \cdot \left( \vec{r}\,' \times \vec{dl}\,' \right) \]


\subsection{Condiciones de contorno}
\label{sub:condiciones_de_contorno}

\[ \vec{n}_{12} \cdot \left.( \vec{J}_2 - \vec{J}_1 ) \right|_S = 0 \]

\subsection{Resistencia}
\label{sub:resistencia}

\subsubsection{Resistencia asociada a un conductor rectangular}
\label{ssub:resistencia_asociada_a_un_conductor_rectangular}

\[ R = \frac{V_A - V_B}{I} = \frac{1}{\sigma} \cdot \frac{L}{S} \quad (\Omega) \]

\section{Campos con variación temporal}
\label{sec:campos_con_variacion_temporal}

\subsection{Ley de Faraday-Lenz}
\label{sub:ley_de_faraday_lenz}

\[ \oint_C \vec{E} \cdot \vec{dl} = - \frac{\partial \phi_B}{\partial t} \]
\[ \phi_B = \iint_S \vec{B} \cdot \vec{dS} \quad ( \mbox{Wb} ) \]

\[ \mbox{f.e.m. } = - \frac{\partial \phi_B}{\partial t} \quad ( \mbox{V} ) \]

\subsection{Coeficientes de Inducción}
\label{sub:coeficientes_de_induccion}

\subsubsection{Coeficiente de autoinducción}
\label{ssub:coeficiente_de_autoinduccion}

\[ L_{i,j} = \frac{\phi_{B\,i,j}}{I_j} \quad ( \mbox{H} ) \]

\section{Extra}
\label{sec:extra}

\subsection{Trigonometría}
\label{sub:trigonometria}

\subsubsection{Teorema del coseno}
\label{ssub:teorema_del_coseno}

\[ c^2 = a^2 + b^2 - 2ab\cos\gamma \]

\subsection{Física}
\label{sub:fisica}

\[ F = - \frac{\partial W}{\partial d} \quad ( \mbox{N} ) \]


\end{document}
